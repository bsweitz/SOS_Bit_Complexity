\section{Introduction}
\label{sec:intro}

Sum of squares (SoS) proof system is a versatile and powerful approach to certifying polynomial inequalities. 
%
SoS certificates can be shown to underly a vast number of algorithms in combinatorial optimization. 
%
On the one hand, SoS certificates hold the promise of yielding algorithms that possibly refute the notorious unique games conjecture \cite{BBHKSZ12, barak2011rounding,GuruswamiS11}.  
%
On the other hand, a flurry of recent works have applied SoS proofs to develop algorithms for problems ranging from constraint satisfaction problems to tensor problems.  

To illustrate sum of squares certificates, let us consider the example of the
\textsc{Balanced Separator} problem.  Here we are given a graph $G = (V,E)$ and the goal is to find a balanced cut $(S,\overline{S})$ with the minimum number of crossing edges.
%
Like many problems in combinatorial optimization, it can be reformulated as a low-degree polynomial optimization problem.  
%
Specifically if we associate $\{0,1\}$ variables $\{x_1,\ldots,x_n\}$ for the vertices of the graph $G$ then we can rewrite the \textsc{Balanced Separator} problem as follows,
\begin{align*}
\text{ Minimize } \ \ \  \sum_{(i,j) \in E} (x_i-x_j)^2  \ \ \ \ \text{ subject to } \left\{ x_i^2 = x_i \forall i\ \ \frac{n}{3} \leq \sum_i x_i \leq \frac{2n}{3} \right\}
\end{align*} 
Here the constraint $x_i^2 = x_i$ ensures $x_i \in \{0,1\}$ while the inequalities enforce the condition that the cut is balanced.

More generally, a low-degree polynomial optimization is of the form,
\begin{align*}
\text{ Minimize } & r(x)   & \text{ subject to } \text{ equalities } \cP = \left\{ p_i(x) = 0 | i \in [n] \right\} \text{ and inequalities } \cQ = \left\{ q_i(x) \geq 0 | i \in [m] \right\}
\end{align*} 

A SoS certificate of a lower bound $ r(x) \geq \theta$ is given by a polynomial identity of the form
\[ r(x) - \theta  = \sum_{i} h_i(x)^2 + \sum_{i \in [m]} \left(\sum_{j}^{t_j} s_j^2(x) \right) \cdot q_j(x) + \sum_{i \in [n]} \lambda_i(x) p_i(x) \]
Notice that for all $x$ satisfying the equalities $\cP$ and the inequalities $\cQ$, the right hand side of the above identity is manifestly non-negative, thereby certifying that $r(x) \geq \theta$.  The degree of the SoS certificate is the maximum degree of the polynomials involved, i.e., $d = \max\{\deg h_i^2, \deg s_j^2 q_i, \deg \lambda_i p_i\}$.

The main appeal of SoS certificates for polynomial optimization is that the existence of a degree $d$ SoS certificate can be formulated as the feasibility of a semidefinite program (SDP).
%
This is the degree $d$ SoS relaxation first introduced by Shor \cite{shor1987class}, and expanded upon by later works of Nesterov \cite{nesterov2000squared}, Grigoriev and Vorobjov \cite{grigoriev2001complexity}, Lasserre \cite{lasserre2000optimisation,lasserre2001global}  and Parrilo \cite{parrilo2000structured}.
(see, e.g., \cite{laurent2009sums,} for many more details).


The degree $d$ SoS SDP has $n^{O(d)}$ variables, and if the coefficients of $p$ and $q$ are reasonably bounded (smaller than $2^{n^{O(d)}}$), the resulting SDP has a compact description of size $n^{O(d)}$.
%
From this, several works including those by the authors, asserted that the resulting feasibility SDP can be solved in time $n^{O(d)}$ using the Ellipsoid algorithm.

In a recent work, O'Donnell \cite{odonnell17} observed that this often repeated claim is far from true.
%
Specifically, O'Donnell exhibited systems of polynomial inequalities with bounded coefficients such that only degree $2$ SoS certificates of non-negativity involve coefficients that are doubly exponential in size.
%
Thus all SoS certificates need an exponential number of bits to represent and consequently, the ellipsoid algorithm will incur an exponential running time.
%
 
As pointed out by O'Donnell, the issue at hand here is not just that of additive error in the solution, i.e., the difference between testing feasibility and near-feasibility.  
%
Indeed, semidefinite programming via the ellipsoid algorithm can only test feasibility up to a very small additive error.
%
However, in a majority of applications of SoS SDP relaxations in combinatorial optimization, the variables in the underlying polynomial system are explictly bounded (also known as Archimedian).
%
Specifically, these include constraints such as $\{ x_i^2 \leq 1 | i \leq [n]\}$, which yield explicit bounds on the values of the variables.
%
In these settings, if there is an approximate SoS certificate for $r(x) \geq \theta$, then there exists a proper SoS certificate for a slightly weaker lower bound $r(x) \geq \theta - o(1)$.
%
Therefore, additive error incurred in semidefinite programming can often be traded off for a slightly weaker objective value.
%
The issue highlighted by O'Donnell is far more serious in that the coefficients of the SoS certificate are too large -- thereby directly affecting the runtime of the ellipsoid algorithm.

On a positive note, O'Donnell shows that a polynomial system whose only constraints are the Boolean constraints $\{x_i^2 = x_i | i \in [n]\}$ always admit SoS certificates with polynomial bit complexity.  
%
He proceeds to ask whether all polynomial systems that include boolean constraints, potentially among others, always admit bounded SoS certificates.


\subsection{Our Results}

In this work, we further explore the issue of bit complexity of SoS proofs, and obtain both positive and negative results.
%

First, we present an easily verifiable and broadly applicable set of sufficient conditions under which a polynomial optimization problem has small SoS certificates.
%
Roughly speaking, we show that polynomial systems with {\it rich} sets of solutions have bounded SoS certificates of non-negativity.
%
More precisely, consider a system consisting of polynomial equalities $\cP$ and inequalities $\cQ$.  Our approach consists of looking for assignments $S$ satisfying three criteria (see \prettyref{def:nice} and \prettyref{thm:main} for formal statements).  
\begin{theorem}
Assume $(\cP, \cQ, S)$ satisfies:
\begin{enumerate}
\item The assignments $S$ robustly satisfy the inequalities in $\cQ$.  
\item The polynomial calculus proof system is both complete and efficient over $S$.  In other words, all degree $d$ polynomial identities over $S$ can be derived using a degree $O(d)$ polynomial derivation from the equalities $\cP$.
\item The assignments $S$ are {\it spectrally rich} in that smallest non-zero eigenvalue of their covariance matrix is at least $2^{-\poly(n^d)}$. 
\end{enumerate}
Then if $r$ has a degree $d$ proof of non-negativity from $\cP$ and $\cQ$, it also has a degree $O(d)$ proof of non-negativity with coefficients bounded by $2^{\poly(n^d)}$.
\end{theorem}

We demonstrate the broad applicability of the above set of sufficient conditions by using them to show upper bounds on bit complexity for \textsc{Max-CSP}, \textsc{Max-Clique}, \textsc{Matching}, \textsc{Balanced Separator}, \textsc{Max-Bisection}, and optimization over the unit sphere.  In each case, the above sufficient conditions can be verified easily. 
%

The above set of sufficient conditions are widely applicable in combinatorial optimization, wherein the polynomial system is typically a relaxation of a well-known set of integer solutions.  
%
In such a setup with integer solutions, we observe in \prettyref{sec:nicespaces} that spectral richness is an immediate consequence of the discrete nature of the set of solutions.
%
Therefore, in all these setups, the only non-trivial thing to verify is the efficiency of the polynomial calculus proof system.
%


The work of O'Donnell \cite{odonnell17} exhibited a polynomial system with bounded coefficients which admitted degree $2$ SoS certificate, whose coefficients were necessarily doubly-exponential.
%
However, the variables in this polynomial system were not all boolean, i.e. did not have the $x_i^2 = x_i$ constraint.
%
In fact, O'Donnell asked whether every polynomial system with boolean constraints admits a small SoS proof.
%
Moreover, the polynomial system in \cite{odonnell17} admits a degree $4$ SoS certificate with small bit complexity.  
%
This opens up the possibility that one can effectively reduce the bit-complexity by raising the degree of the proof.
%
For instance, if a system admits a degree $d$ SoS certificate then does it always admit a degree $2^d$ SoS certificate with small bit complexity (even under boolean constraints)?
Unfortunately, we refute both of the above possibilities by exhibiting a counterexample.
%
Formally, we show the following:

\begin{theorem}\label{thm:counter}
	There exists a system of quadratic equations on $n$ variables such that
	\begin{itemize}
	\item The system includes the equation $x_i^2 - x_i = 0$ for each $i \in [n]$.
	\item There exists a polynomial with a degree $2$ SoS certificate of non-negativity, albeit with doubly exponentially large coefficients.
	\item No SoS certificate of degree $d \leq \sqrt{n}$ has coefficients smaller than $\Omega\left(\frac{1}{n^d}\cdot 2^{\exp(\sqrt{n})}\right)$.
	\end{itemize}	
\end{theorem}

********** todo
**** consistent with terminology of nice solutions, 
**** be consistent with terminology, polynomial system, polynomial inequalities..
**** check if above theorem statement agrees with final result
**** mark above theorem as informal, or use theoremrestate 
**** Broken References



