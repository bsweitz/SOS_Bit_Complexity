\section{Effective Nullstellensatz Yields Low Bit Complexity}
\label{sec:main}

In this section we prove our main theorem:

\begin{theorem}\label{thm:main}
Let $\cP = \{p_1,\dots,p_m\}$ be a real complete, $(k(d),B)$-effective set of polynomials with $S = \cV_\R(\cP)$. Let $r(x)$ be a polynomial nonnegative on $S$, and assume $r$ has a degree $d$ sum-of-squares proof of nonnegativity $r(x) = \sum_{i=1}^t q_i^2 + \sum_{i=1}^m \lambda_i p_i$. Then $r$ has a degree $k(d)$ sum-of-squares proof of nonnegativity such that the coefficients of every polynomial appearing in the proof are bounded by $O(\exp(\|\cP\| + \|r\|))$.
\end{theorem}
\begin{proof}
Our strategy is to simplify the SOS part of the given SOS-proof and move all of the potentially huge coefficients into the latter term which lies in the ideal $I = \langle \cP\rangle$. Next, we use the fact that $\cP$ is effective to argue that you don't need to use huge coefficients to express the leftover polynomial. To that end, we want to reduce each $q_i$ to a canonical form. Recall that since $\cP$ is real complete, $V = \R[x]/I$ is a real vector space with inner product $\iprod{p, q} = \sum_{\alpha \in S} p(\alpha)q(\alpha)$. Let $V_d = (\R[x]/I)_d$ denote the subspace of $V$ where each equivalence class contains a representative of degree at most $d$. Clearly the standard monomial basis is a spanning set for $V_d$, so it contains some basis for $V_d$. Take this basis and the defined inner product and use the Gram-Schmidt process to produce an orthonormal basis $\{v_1,\dots,v_s\}$ for $V_d$. We need to bound $\|v_i(x)\|$: Note that each $\alpha \in S$ satisfies $\|\alpha\| = \poly(\|\cP\|)$, and so at each step in the Gram-Schmidt process, the inner product between two polynomials $p$ and $q$ is at most $\poly(\|\cP\|,\|p\|,\|q\|,n^d)$, and since there are at most $n^d$ steps, it produces polynomials satisfying $\|v_i(x)\| \leq \poly(\|cP\|,n^d)$ for each $i$. Define a vector of polynomials $v = [v_1,\dots,v_s]$, and note there must exist vectors of reals $c_1,\dots,c_t$ such that $\forall \alpha \in S: c_i^Tv(\alpha) = q_i(\alpha)$, and thus $q_i^2(x) - (c_i^Tv(x))^2 \in I$. Thus we can write $r$ in the following way:

\begin{align*}
r(x) &= \sum_{i=1}^t (c_i^Tv(x))^2 + \sum_{i=1}^t \left(q_i^2(x) - (c_i^Tv(x))^2\right) + \sum_{i=1}^m \lambda_ip_i \\
&= \sum_{i=1}^t c_ic_i^T \cdot v(x)v(x)^T + \sum_{i=1}^t \left(q_i^2(x) - (c_i^Tv(x))^2\right) + \sum_{i=1}^m \lambda_ip_i \\
&= (C \cdot v(x)v(x)^T) + \sum_{i=1}^t \left(q_i^2(x) - (c_i^Tv(x))^2\right) + \sum_{i=1}^m \lambda_ip_i
\end{align*}

Now if we take this polynomial identity and average it over every $\alpha \in S$, we get
\begin{align*}
E_\alpha[r(\alpha)] &= (C \cdot E_\alpha[v(\alpha)v(\alpha)^T]) + 0 + 0\\
&= C \cdot Id \\
&= Tr(C)
\end{align*}
where we used the fact that $\{v_1,\dots,v_s\}$ are orthonormal. Clearly $E_\alpha[r(\alpha)] = \poly(\|r\|)$, and since $C$ is PSD, this implies $\|C\|_F = \poly(\|r\|)$, and thus $\|v(x)^TCv(x)\| = \poly(\|r\|,\|\cP\|,n^d)$. Now because $\cP$ is complete, $r(x) - (C \cdot v(x)v(x)^T) \in \langle \cP\rangle$. Because $\langle \cP\rangle$ is $(k(d), B)$-effective, there is a degree $k(d)$ polynomial identity
\[r(x) - (C \cdot v(x)v(x)^T) = s(x) + \sum_i \lambda'_i(x)p_i(x)\]
where $s(x)$ is a sum-of-squares polynomial with coefficients bounded by $B$. If we let the coefficients of polynomials $\sigma_i$ be variables, it's clear that the system of equations induced by the polynomial identity 
\[r(x) - (C\cdot v(x)v(x)^T) - s(x) = \sum_i \sigma_i(x)p_i(x)\]
is feasible, but note that this system is \emph{linear} in the coefficients of $\sigma_i$, there are at most $O(n^{k(d)})$ equations, and the entries are at most $\poly(\|r\|,\|\cP\|,n^d, B)$. By Cramer's rule, we can pick a solution $\sigma_i^*$ with $\|\sigma_i^*\| \leq \exp(\poly(\|r\|,\|\cP\|,n^d, B))$. Thus we can replace the $\lambda_i'$ with $\sigma_i^*$ and achieve an SOS-proof of bounded bit complexity. 
\end{proof}
We obtain an immediate corollary:
\begin{corollary}\label{cor:grobner}
If $\cP$ is complete and a Grobner basis for $\langle \cP\rangle$, then any degree $d$ SOS proof from $\cP$ can be taken to have polynomial bit complexity. 
\end{corollary}
\begin{proof}
Every Grobner basis is $d$-effective, so just apply \prettyref{thm:main}.
\end{proof}
\begin{remark}
In CITE ODONNEL, the author gives an example $\cP$ and linear $r(x)$ for which $r$ has a degree two SOS proof, but it must have exponential bit-complexity. We simply note here that the $\cP$ he gives is $(d+2)$-effective but not $d$-effective, so if you're willing to pay a tiny bit in the degree you can take the SOS proofs to have polynomial bit-complexity.
\end{remark}
With our main theorem in hand, we move on to using it to prove that many applications of the SOS algorithm remain automatizable. One might worry that Grobner bases are the \emph{only} types of base constraints that are effective, and that this theorem will not be very useful. We dispel this notion in \prettyref{sec:balance} by showing the natural description of the MAX-BISECTION ideal is effective, even though its Grobner basis has exponential size. Before we move on to these though, we strengthen the theorem slightly. Note that we are only allowing ourselves efficient Nullstellensatz-type proofs, even though the SOS proof system is more powerful. As an example, consider the following system: $\cP = \{x_i^2 - 1 | i \in [n]\} \cup \{\sum_i x_i - n\}$. It's clear that the only feasible solutions to this system are $x_i = 1$ for all $i$. However, the polynomial $x_i - 1$ does not have a derivation in degree less than $n$ from these constraints! However, if we allow ourselves to use SOS-proofs, one can write both $x_i - 1$ and $1 - x_i$ as a SOS modulo $\cP$, thereby proving that $x_i - 1 \in I$. This motivates the following (obvious) lemma:
\begin{lemma}\label{lem:sos-lemma}
Let $\cP$ and $\cQ$ be such that $\langle \cP\rangle = \langle \cQ\rangle$ and let $\cQ$ be complete and $k(d)$-effective. If every $q \in \cQ$ of degree $d$ can be written $q(x) = \sum_i s_i(x)^2 + \sum_i \lambda_i(x)p_i(x)$ in degree $k'(d)$ with $\|s_i\|, \|\lambda_i\| \leq poly(\|\cP\|)$, then for any SOS proof of $r(x)$ from $\cP$ of degree $d$, there is an SOS proof of $r(x)$ from $\cP$ of degree $k(d) + k'(d)$ with coefficients bounded by $O(\exp(\|\cP\| + \|r\|))$.
\end{lemma}
