\begin{abstract}
In \cite{} the author notes that the long-accepted truth that $d$ rounds of the SOS algorithm can be implemented in $\poly(n^d)$ time is not necessarily true. He presents an example of a polynomial system which contains low-degree proofs of non-negativity, but these proofs necessarily involve numbers with an exponential number of bits, which causes the Ellipsoid algorithm to fail. In this paper we present some of the first theory behind bounding the magnitudes of coefficients in SOS proofs. We prove that if the solution space underlying the polynomial optimization problem is nice enough, then any SOS proofs from the constraints can be taken to have polynomial bit length. Our condition captures common use-cases for the SOS algorithm, such as CSPs, BALANCED-CSPs, CLIQUE, and MAX-BISECTION constraints. The author of \cite{} also conjectured that perhaps any polynomial system containing Boolean constraints could have proofs of polynomial bit complexity. We answer this question in the negative, giving a counterexample system and non-negative polynomial which has degree two proofs, but no proof with small coefficients until degree $\Omega(\sqrt{n})$.
\end{abstract}