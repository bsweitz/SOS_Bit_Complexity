\begin{abstract}
In \cite{odonnell17}, Ryan O'Donnell notes that the long-accepted result that $d$ rounds of the Sum-of-Squares algorithm can be implemented in $\poly(n^d)$ time is not necessarily true. He presents an example of a polynomial system which contains low-degree proofs of non-negativity, but these proofs necessarily involve numbers with an exponential number of bits, which causes the Ellipsoid algorithm to take exponential time. In this paper we present some of the first theory behind bounding the magnitudes of coefficients in SoS proofs. We prove that if the solution space underlying the polynomial optimization problem is nice enough, then any SoS proofs from the constraints can be taken to have polynomial bit length. Our condition captures common use-cases for the SoS algorithm, such as \textsc{Max-CSP}, \textsc{Balanced Separator}, \textsc{Max-Clique}, \textsc{Max-Bisection}, and \textsc{Unit-Vector} constraints. O'Donnell hoped that perhaps any polynomial system containing Boolean constraints could have proofs of polynomial bit complexity. We answer this question in the negative, giving a counterexample system and non-negative polynomial which has degree two proofs, but no proof with small coefficients until degree $\Omega(\sqrt{n})$.
\end{abstract}