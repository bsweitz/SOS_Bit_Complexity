\section{Examples with Nice Solution Spaces}
\label{sec:nicespaces}
In this section we discuss what kinds of polynomial constraints fit into our classification of having nice solution spaces. Throughout this section we will take $\cB$ to be the standard monomial basis of $\R[x]_d$. 
\subsection{Solution Spaces in the Boolean Hypercube}
First, we consider the case $S \subseteq \{0,1\}^n$. In this case, it is relatively simple to prove spectral richness.
\begin{lemma}\label{lem:integer-rich}
Let $(\cP, \cQ, S)$ be such that $S \subseteq \{0,1\}^n$. Then $(\cP, \cQ, S)$ is $\delta$-spectrally rich with $\frac{1}{\delta} = O(\poly(2^{n^d}))$.
\end{lemma}
\begin{proof}
Define $M$ as before, and note that $2^nM$ is an integer matrix with entries at most $2^n$. Let $A$ be a full-rank principal minor of $2^nM$ and WLOG let it be the upper-left block of $M$. We claim the least eigenvalue of $A$ lower bounds the least nonzero eigenvalue of $M$. Because $M$ is symmetric, there must be a $B$ such that
\[M = \left[\begin{tabular}{c} $I$ \\ $B$\end{tabular}\right]A\left[\begin{tabular}{cc} $I$ & $B^T$\end{tabular}\right].\]
Let $P = [I, B^T]$, $\rho$ be the least eigenvalue of $A$, and $x$ be a vector perpindicular to the zero eigenspace of $M$. Then we have $x^TMx \geq \rho x^TP^TPx$,
but $x$ is also perpindicular to the zero eigenspace of $P^TP$. Now $P^TP$ has the same nonzero eigenvalues as $PP^T = I + B^TB \succeq I$, and thus $x^TP^TPx \geq 1$, and so every nonzero eigenvalue of $M$ is at least $\rho$. Now $A$ is a full-rank bounded integer matrix with dimension at most $n^d$. It has determinant at least $1$ and eigenvalues at most $2^nn^d$, so its least eigenvalue must be at least $(2^nn^d)^{-n^d}$.
\end{proof}
A similar proof can be used whenever the solution space is discrete and separated by some constant. 

To prove completeness, we typically want to show two things. First, that every degree $d$ polynomial in $\langle \cP\rangle$ has a degree at most $k$ derivation. Second, that there are no polynomials outside $\langle \cP\rangle$ that are zero on $S$. This second condition can be thought of as saying that $\cQ$ is somehow nondegenerate. If there are extra polynomial equalities implied by $\cQ$, they should be included in $\cP$. The following is a very simple case for which this is true:
\begin{lemma}\label{lem:grobner}
If $\cQ = \emptyset$ and $\cP$ is a Grobner basis for $\langle \cP\rangle$, then $(\cP,\cQ,S)$ is $d$-complete up to degree $d$. 
\end{lemma}
\begin{proof}
If $\cP$ is a Grobner basis, then every degree $d$ polynomial in $\langle \cP\rangle$ has a degree $d$ derivation via multivariate division. Because $\cQ = \emptyset$, the polynomials that are zero on $S$ are exactly the polynomials in $\langle \cP\rangle$. 
\end{proof}
Two polynomial optimization problems captured by \prettyref{lem:grobner} are maximizing polynomials subject to only Boolean constraints (for example, CSP's), and subject to the CLIQUE constraints. While this fact was already known for SOS proofs from Boolean constraints, our technique is the first one to give results for systems with additional constraints. 

Furthermore, if $\cQ$ is pretty simple, it is not too difficult to prove that it does not introduce any additional low degree polynomial equalities. For example, consider the BALANCED-CUT constraints: $\cP = \{x_i^2-x_i | i \in [n]\}$ and $\cQ = \{\sum_i x_i - n/3, 2n/3 - \sum_i x_i\}$. These have a solution space on the bit strings with between $n/3$ and $2n/3$ ones. The polynomial equalities that define this set are the Boolean equalities, $\prod_{i \in T} x_i$, $\prod_{i \in T} (1-x_i)$ for each $T: |T| > 2n/3$. These are the smallest degree polynomials that are zero on $S$ that are not in $\langle \cP\rangle$, and $\cP$ is a Grobner basis, so $(\cP, \cQ, S)$ is $d$-complete as long as $d < 2n/3$. The polynomials in $\cQ$ can be perturbed by $1/2$ to make them $1/2$-robust, so these constraints are nice. 

One might worry that Grobner bases are the \emph{only} constraints that are nice, and that \prettyref{thm:main} will not be very applicable. We dispel this notion in \prettyref{sec:balance} by showing the natural description of the MAX-BISECTION ideal is complete, even though its Grobner basis has exponential size.

\subsection{Sphere Constraint}
Let $\cP = \{\sum_i x_i^2 - 1\}$ and $\cQ = \emptyset$. Then $S = \{x: \|x\| = 1\}$, and $E_{\alpha \in S}[{\bf v}{\bf v}^T] = $

We collect the examples discussed in this section here for simplicity:
\begin{theorem}\label{thm:examples}
The following constraints are nice with $\delta = ()$, $k = ()$, and $\epsilon = ()$:
\begin{itemize}
\item CSPs: $\cP = \{x_i^2 - x_i | i \in [n]\}$. 
\item CLIQUE: $\cP = \{x_i^2 - x_i | i \in [n]\} \cup \{x_ix_j | (i,j) \in E\}$.
\item MATCHING: $\cP = \{x_{ij}^2 - x_{ij} | i,j \in [n]\} \cup \{\sum_i x_{ij} - 1 | i \in [n]\} \cup \{x_{ij}x_{ik} | i,j,k \in [n]\}$.
\item BALANCED CSPs: $\cP = \{x_i^2 - x_i | i \in [n]\}$, $\cQ = \{2n/3 - \sum_i x_i, \sum_i x_i - n/3\}$.
\item MAX-BISECTION: $\cP = \{x_i^2 - x_i | i \in [n]\} \cup \{\sum_i x_i - n/2\}$.
\item UNIT-VECTOR: $\cP = \{\sum_i x_i^2 - 1\}$.
\end{itemize}
\end{theorem}

\subsection{Limitations}
While \prettyref{thm:main} allows us to prove that many different systems of polynomial constraints have well-behaved SOS proofs, there are a few areas where it comes up short. Most noticeably, our theorem requires that there by a sufficiently rich set of solutions to the polynomials. If there are no solutions to $(\cP, \cQ)$, it is not applicable. This comes up when trying to use SOS to refute a system of infeasible polynomial equations. For example, one common technique is to introduce lower bounds on the objective function $f(x)$ of a maximization problem as constraints and attempt to use SOS to find a refutation, i.e. a proof of non-negativity for the constant polynomial $-1$. We are unable to show that these proofs can be taken to have polynomial bit complexity since they have empty solution spaces. As another example, we are unable to use our framework to show that refutations of the knapsack constraints use only polynomially many bits, even though it is clear by simply examining these known refutations that they only involve small coefficients. 