\section{Examples with Nice Solution Spaces}
\label{sec:nicespaces}
In this section we discuss what kinds of polynomial constraints fit into our classification of having nice solution spaces.
\subsection{Solution Spaces in the Boolean Hypercube}
First, we consider the case $S \subseteq \{0,1\}^n$. In this case, it is relatively simple to prove spectral richness.
\begin{lemma}\label{lem:integer-rich}
Let $\cP$ and $\cQ$ be such that $S \subseteq \{0,1\}^n$. Then $S$ is $\delta$-spectrally rich with $\frac{1}{\delta} = 2^{\poly(n^d)}$.
\end{lemma}
\begin{proof}
Recall $M = E_{\alpha \in S}[{\bf v}(\alpha){\bf v}(\alpha)^T]$, and note that $2^nM$ is an integer matrix with entries at most $2^n$. Let $A$ be a full-rank principal minor of $2^nM$ and WLOG let it be the upper-left block of $M$. We claim the least eigenvalue of $A$ lower bounds the least nonzero eigenvalue of $M$. Because $M$ is symmetric, there must be a $B$ such that
\[M = \left[\begin{tabular}{c} $I$ \\ $B$\end{tabular}\right]A\left[\begin{tabular}{cc} $I$ & $B^T$\end{tabular}\right].\]
Let $P = [I, B^T]$, $\rho$ be the least eigenvalue of $A$, and $x$ be a vector perpindicular to the zero eigenspace of $M$. Then we have $x^TMx \geq \rho x^TP^TPx$,
but $x$ is also perpindicular to the zero eigenspace of $P^TP$. Now $P^TP$ has the same nonzero eigenvalues as $PP^T = I + B^TB \succeq I$, and thus $x^TP^TPx \geq 1$, and so every nonzero eigenvalue of $M$ is at least $\rho$. Now $A$ is a full-rank bounded integer matrix with dimension at most $n^d$. It has determinant at least $1$ and eigenvalues at most $2^nn^d$, so its least eigenvalue must be at least $(2^nn^d)^{-n^d}$. A similar proof can be used whenever the solution space is discrete and separated by some constant. 
\end{proof}

To prove completeness, we typically want to show two things. First, that every degree $d$ polynomial in $\langle \cP\rangle$ has a degree at most $k$ derivation. Second, that there are no polynomials outside $\langle \cP\rangle$ that are zero on $S$. This second condition can be thought of as saying that $\cQ$ is somehow nondegenerate. If there are extra polynomial equalities implied by $\cQ$, they should be included in $\cP$. The simplest case is when $\cQ$ is empty, and $\cP$ is something called a Gr\"obner basis. A Gr\"obner basis for an ideal is a generating set of polynomials that allow a well-defined multivariate polynomial division. Computing the Gr\"obner basis is frequently the first step in practical polynomial equation solvers, and we note the following easy lemma:
\begin{lemma}\label{lem:grobner}
If $\cQ = \emptyset$ and $\cP$ is a Grobner basis for $\langle \cP\rangle$, then $S$ is $d$-complete up to degree $d$. 
\end{lemma}
\begin{proof}
If $\cP$ is a Gr\"obner basis, then every degree $d$ polynomial in $\langle \cP\rangle$ has a degree $d$ derivation via multivariate division. Because $\cQ = \emptyset$, the polynomials that are zero on $S$ are exactly the polynomials in $\langle \cP\rangle$. 
\end{proof}
Two polynomial optimization problems captured by \prettyref{lem:grobner} are the \textsc{Max-CSP} problem, and the \textsc{Max-Clique} problem. While this fact was already known for \text{Max-CSP}, our technique is the first one to give results for systems with more than just Boolean constraints.

Furthermore, if $\cQ$ is pretty simple, it is not too difficult to prove that it does not introduce any additional low degree polynomial equalities. For example, consider \textsc{Balanced Separator}: $\cP = \{x_i^2-x_i | i \in [n]\}$ and $\cQ = \{\sum_i x_i - n/3, 2n/3 - \sum_i x_i\}$. These have a solution space $S$ on the bit strings with between $n/3$ and $2n/3$ ones. Now if $r$ is a nonzero multilinear polynomial which is zero on $S$, then its symmetrized version $sr = \frac{1}{n!}\sum_{\sigma \in \cS_n} \sigma r$ must also be zero on $S$, where $\sigma$ acts by permuting the variable names. However, $sr$ is a univariate polynomial in $\sum_i x_i$ (modulo the Boolean constraints). This univariate polynomial has $n/3$ zeros, and thus must have degree at least $n/3$. Since symmetrizing doesn't change degree, we conclude that $r$ also has degree at least $n/3$. Thus the polynomials that are zero on $S$ that aren't in $\langle \cP\rangle$ have degree at least $n/3$. Further, $\cP$ is a Gr\"obner basis, so $(\cP, \cQ, S)$ is $d$-complete as long as $d < n/3$. The polynomials in $\cQ$ can be perturbed by $1/2$ to make them $1/2$-robust, so $(\cP, \cQ, S)$ is nice. 

Gr\"obner bases are useful, but often extremely expensive to compute, and having access to one is a luxury in many optimization problems. Because of this, one might worry that Gr\"obner bases are the \emph{only} constraints that are nice, and that \prettyref{thm:main} will not be very applicable. We dispel this notion in \prettyref{sec:balance} by showing the natural description of the \textsc{Max-Bisection} ideal is $d$-complete, even though its Gr\"obner basis has exponential size. Likewise, in \cite{Braun:2016:MPN:2884435.2884510}, it was proven that the \textsc{Matching} constraints are $2d$-complete, which is another set of constraints that is not a Gr\"obner basis. 

\subsection{\textsc{Unit-Vector} Constraint}
Let $\cP = \{\sum_i x_i^2 - 1\}$, $\cQ = \emptyset$, and thus $S = \{x: \|x\| = 1\}$. This constraint appears frequently in tensor norm problems as a way to enforce scaling. The following is an easy lemma:
\begin{lemma}
Let $(\cP, \cQ, S)$ be as above. Then it is nice. 
\end{lemma}
\begin{proof}
Since $\cQ = \emptyset$, it is clearly robust. It may be well-known that $\cP$ is $d$-complete, but we could not find a reference so we record it here for completeness. Let $p(x)$ be any degree $d$ polynomial which is zero on the unit sphere, and define $p_0(x) = p(x) + p(-x)$. Clearly $p_0$ is also zero on the unit sphere, with degree $k = 2\lfloor (d+1)/2 \rfloor$. Note that $p_0$ has only terms of even degree. Define $q_i$ to be the part of $p_i$ which has degree strictly less than $k$, and let $p_{i+1} = p_i + q_i\cdot(\sum_i x_i^2 - 1)$. Then each $p_i$ is zero on the unit sphere and has no monomials of degree strictly less than $2i$. Thus $p_{k/2}$ is homogeneous of degree $k$. But then $p(tx) = t^kp_k(x) = 0$ for any unit vector $x$ and $t > 0$, and thus $p_k(x)$ must be the zero polynomial. This implies that $p_0$ is a multiple of $\sum_i x_i^2 - 1$. The same logic shows that $p(x) - p(-x)$ is also a multiple of $\sum_i x_i^2 - 1$, and thus so is $p(x)$. Now $\langle \cP\rangle$ is principal, so every degree $d$ polynomial in it has a degree $d$ derivation, so $(\cP, \cQ, S)$ is $d$-complete.

To prove spectral-richness, we note that in \cite{10.2307/2695802} the author gives an exact formula for each entry of the matrix $M = \int_{S} p(x)$ for any polynomial $p$. The formulas imply that $(n+d)!\pi^{-n/2} M$ is an integer matrix with entries (very loosely) bounded by $(n+d)!d!2^n$. By the same proof as in \prettyref{lem:integer-rich}, we conclude that $(\cP, \cQ, S)$ is $\delta$-spectrally rich with $1/\delta = 2^{\poly(n^d)}$.
\end{proof}

We collect the examples discussed in this section here for simplicity:
\begin{corollary}\label{cor:examples}
The following constraints are nice:
\begin{itemize}
\item \textsc{Max-CSP}: $\cP = \{x_i^2 - x_i | i \in [n]\}$. 
\item \textsc{Max-Clique}: $\cP = \{x_i^2 - x_i | i \in [n]\} \cup \{x_ix_j | (i,j) \in E\}$.
\item \textsc{Balanced Separator}: $\cP = \{x_i^2 - x_i | i \in [n]\}$, $\cQ = \{2n/3 - \sum_i x_i, \sum_i x_i - n/3\}$.
\item \textsc{Max-Bisection}: $\cP = \{x_i^2 - x_i | i \in [n]\} \cup \{\sum_i x_i - n/2\}$.
\item \textsc{Matching}: $\cP = \{x_{ij}^2 - x_{ij} | i,j \in [n]\} \cup \{\sum_i x_{ij} - 1 | i \in [n]\} \cup \{x_{ij}x_{ik} | i,j,k \in [n]\}$.
\item \textsc{Unit-Vector}: $\cP = \{\sum_i x_i^2 - 1\}$.
\end{itemize}
\end{corollary}

\subsection{Limitations}
While \prettyref{thm:main} allows us to prove that many different systems of polynomial constraints have well-behaved SoS proofs, there are a few areas where it comes up short. Most noticeably, to contain a blocking set of solutions the solution space has to be nonempty. This can be a problem when trying to find SoS proofs of infeasibility. For example, one common technique is to introduce lower bounds on an objective function $f(x)$ of a maximization problem as constraints and attempt to use SoS to find a refutation, i.e. a proof of non-negativity for the constant polynomial $-1$. We are unable to show that these proofs can be taken to have polynomial bit complexity since they have empty solution spaces. As another example, we are unable to use our framework to show that refutations of the knapsack constraints use only polynomially many bits, even though it is clear by simply examining these known refutations that they only involve small coefficients. 