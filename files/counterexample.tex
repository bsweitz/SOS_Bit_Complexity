\section{Boolean Systems With No Small-Coefficient Proofs}\label{sec:counterexample}
In \cite{odonnell17}, the author gives an example of a polynomial system for which degree two SoS proofs can certify non-negativity of a certain poylnomial, but the proofs necessarily involves coefficients of doubly-exponential size. However, there are two weaknesses in his example system. First, it is not a Boolean one, i.e. it contains variables $y_i$ for which the constraint $y_i^2 - y_i = 0$ is not present in the constraints. Many practical optimization problems have Boolean constraints, and in \cite{odonnell17}, the author hoped that having those constraints might suffice to imply that all proofs could have small bit complexity. Second, while the degree two proofs must have exponential bit complexity, there were degree four proofs of non-negativity with polynomial bit complexity. Thus simply looking for proofs of a little bit higher degree was enough to defeat his counterexample. In this section, we strengthen his counterexample, giving an example of a Boolean system with $n$ variables for which there is a polynomial that has a degree two proof of non-negativity, but no proof with polynomial bit complexity until degree $\Omega(\sqrt{n})$.

\subsection{A First Example}
The original example given in \cite{odonnell17} essentially contains the following system whose repeated squaring is responsible for the blowup of the coefficients in the proofs:
\[\begin{tabular}{ccccc}
$y_1^2 - y_2 = 0$, & $y_2^2 - y_3 = 0$, & $\dots$, & $y_{n-1}^2 - y_n = 0$, & $y_n^2 = 0$.
\end{tabular}\]
It is obvious to mathematicians that the only solution for this system is $(0,0,0,\dots,0)$, so the polynomial $\epsilon - y_1$ must be non-negative over the solution space for any $\epsilon > 0$. It is not as obvious whether or not an SoS proof of this non-negativity exists. It turns out that there is a degree two SoS proof as follows:
\begin{align}
\epsilon - y_1 &\equiv \left(\sqrt{\frac{\epsilon}{n}} - \left(\frac{n}{4\epsilon}\right)^{1/2}y_1\right)^2 + \left(\sqrt{\frac{\epsilon}{n}} - \left(\frac{n}{4\epsilon}\right)^{3/2}y_2\right)^2 + \left(\sqrt{\frac{\epsilon}{n}} - \left(\frac{n}{4\epsilon}\right)^{7/2}y_3\right)^2 + \nonumber\\
&+\dots + \left(\sqrt{\frac{\epsilon}{n}} - \left(\frac{n}{4\epsilon}\right)^{(2^n-1)/2}y_n\right)^2.\label{eq:proof}\tag{$*$}
\end{align}
where the $\equiv$ is hiding adding multiples of the constraints. Of course, this proof involves coefficients of doubly-exponential size, but one can prove that they are required. We will take $\epsilon < 1/2$ for simplicity. We will define a linear functional $\phi: \R[Y]_d \rightarrow \R$ satisfying the following:
\begin{itemize}
\item $\phi[\epsilon - y_1] = -\epsilon$
\item $\phi[p^2] \geq 0$ for any $p^2$ of degree at most $d$
\item $\phi[\sigma_i(y_i^2 - y_{i+1})] = 0$ for any $i \leq n-1$ and $\sigma_i$ of degree at most $d-2$
\item $|\phi[\lambda y_n^2]| \leq (2\epsilon)^{2^{n-1}}n^d\|\lambda\|$.
\end{itemize}
If such a $\phi$ exists, then for any degree $d$ SoS proof of non-negativity
\[\epsilon - y_1 = \sum_i h_i(y)^2 + \sum_{i=1}^{n-1} \sigma_i(y_i^2 - y_{i+1}) + \lambda \cdot y_n^2,\]
apply $\phi$ to both sides. We obtain $-\epsilon \leq P + 0 + \phi[\lambda y_n^2]$, where $P \geq 0$. Because $|\phi[\lambda y_n^2]| \leq (2\epsilon)^{2^{n-1}}n^d\|\lambda\|$, $\lambda$ must contain a coefficient of size at least $\Omega(\frac{1}{n^d}\left(\frac{1}{2\epsilon}\right)^{2^n})$.

To show such a $\phi$ exists, we define it as follows. By the constraints, every monomial is equivalent to some power of $y_1$. For example, $y_1y_2y_3 \equiv y_1^7$. Let $\phi$ map each monomial to $(2\epsilon)$ raised to this exponent. This is like setting $\phi[y_1] = 2\epsilon$, and $\phi[y_i] = (2\epsilon)^{2^{i-1}}$. One can easily check that this $\phi$ satisfies the above. Unfortunateley, none of the variables in this system are Boolean, so this comprises a fairly weak counterexample. However, we can modify it to get our stronger counterexample.
 
\subsection{A Boolean System}
One simple way to try to make the system Boolean is to just add the constraints $y_i^2 = y_i$ to the system. Unfortunately, in that case it is easy to prove that $y_i - y_j = 0$ for each $i$ and $j$, and of course $y_n = y_n^2 = 0$. It is too easy for SoS to figure out what each $y_i$ should look like. Previously, the variables were unconstrained in any way, and we want to imitate that. We draw inspiration from the Knapsack problem, and we instead replace each instance of the variable $y_i$ with a sum of $2k$ Boolean variables 
\[y_i \rightarrow \sum_j w_{ij} - k,\] 
and we consider the non-negative polynomial $\epsilon - (\sum_j w_{1j} - k)$. Clearly there is a degree two proof of non-negativity for this polynomial since we can just replace each instance of $y_i$ with $\sum_j w_{ij} - k$ in (\ref{eq:proof}). 

It remains to show that there are no other proofs that have only small coefficients. Here, we use the fact that the Knapsack problem is hard for SoS: there is no SoS proof of degree less than $\Omega(k)$ that $\sum_j w_{ij} - k$ is not equal to any number $r \in (0,1)$ \cite{Grigoriev2001}. This allows us to use the Knapsack pseudodistribution to "pretend" that $\sum_j w_{ij} - k = (2\epsilon)^{2^{i-1}}$. Specifically, for each $r \in (0,1)$, there is a linear functional $\phi_r$ defined on polynomials of $2k$ Boolean variables which satisfies
\begin{itemize}
\item $\phi_r[\sigma_{ij}(w_{ij}^2 - w_{ij})] = 0$ for any $\sigma_{ij}$ up to degree $O(k)$
\item $\phi_r[\lambda\cdot((\sum_j w_{ij} - k) - r)] = 0$ for any polynomial $\lambda$ up to degree $O(k)$
\item $\phi_r[p^2] \geq 0$ for any polynomial $p^2$ of degree at most $O(k)$.
\end{itemize}
Now, take the linear functional $\Phi$ defined on each polynomials of $2kn$ variables defined in the following way: Let $T = T_1 \cup T_2 \cup \dots \cup T_n$ where $T_i$ is a multiset that contains only the variables corresponding to $y_i$, and let $w_T$ denote the associated monomial. Then define
\[\Phi[w_T] = \phi_{2\epsilon}(w_{T_1})\phi_{(2\epsilon)^2}(w_{T_2})\dots\phi_{(2\epsilon)^{2^{n-1}}}(w_{T_n}).\]
Clearly $\Phi$ is non-negative on squares and $\Phi[\sigma_{ij}(w_{ij}^2-w_{ij})] = 0$ for any $\sigma_{ij}$ up to degree $\Omega(k)$. Because $\Phi[\lambda(\sum_j w_{ij} - k)] = \Phi[(2\epsilon)^{2^{i-1}}\lambda]$, $\Phi$ also satisfies $\Phi[\lambda((\sum_j w_{ij}-k)^2 - (\sum_j w_{i+1,j} - k))] = 0$ for each $\lambda$ and $1 \leq i \leq n-1$. Finally, because each variable is Boolean, $\Phi$ of any monomial is at most one, so for any monomial $w_M$, $\Phi[w_M(\sum_j w_{nj} - k)^2] = \Phi[(2\epsilon)^{2^{n-1}} w_M] \leq (2\epsilon)^{2^{n-1}}$. There are at most $(nk)^d$ monomials, so $\Phi[\lambda(\sum_j w_{nj} - k)^2] \leq (nk)^d(2\epsilon)^{2^{n-1}}\|\lambda\|$. Just as before, the existence of $\Phi$ implies that any degree $d$ proof of non-negativity for $\epsilon - (\sum_j w_{1j} - k)$ must contain coefficients of size at least $\Omega(\frac{1}{(nk)^d} \cdot \left(\frac{1}{2\epsilon}\right)^{2^n})$. If we set $k = n$, then there are $n^2$ variables and no proof of non-negativity with coefficients smaller than doubly-exponential until degree $n$. This proves \prettyref{thm:counter}.

