\section{Boolean Systems With No Simple Proofs}
In \cite{odonnel}, the author gives an example of a polynomial system for which degree two SOS proofs can certify non-negativity of a certain poylnomial $p$, but the proof necessarily involves coefficients of doubly-exponential size. However, there are two shortcomings in his counterexample. First, his system is not a Boolean one, and second, degree four SOS proofs can certify $p$ with small coefficients. Indeed, at the end of \cite{odonnel}, he poses the question of whether Boolean constraints are sufficient for proofs with small coefficients as an open problem. In this section, we answer his open problem in the negative, giving an example of a Boolean system for which there is a polynomial $q$ that has a degree two proof of non-negativity, but no proof with small coefficients until degree $\Omega(\sqrt{n})$.

The counterexample given in \cite{odonnel} essentially contains the following system which is responsible for the blowup of the coefficients:
\[\begin{tabular}{ccccc}
$y_1^2 = y_2$, & $y_2^2 = y_3$, & $\dots$, & $y_{n-1}^2 = y_n$, & $y_n^2 = 0$.
\end{tabular}\]
It is obvious to mathematicians that the only solution for this system is $(0,0,0,\dots,0)$. Indeed, there is a degree two SOS proof of any arbitrarily small positive upper bound on $y_1$:
\begin{align*}
\epsilon - y_1 &\equiv \left(\sqrt{\frac{\epsilon}{n}} - \left(\frac{n}{4\epsilon}\right)^{1/2}y_1\right)^2 + \left(\sqrt{\frac{\epsilon}{n}} - \left(\frac{n}{4\epsilon}\right)^{3/2}y_2\right)^2 + \left(\sqrt{\frac{\epsilon}{n}} - \left(\frac{n}{4\epsilon}\right)^{7/2}y_3\right)^2 + \\
&+\dots + \left(\sqrt{\frac{\epsilon}{n}} - \left(\frac{n}{4\epsilon}\right)^{(2^n-1)/2}\right)^2.
\end{align*}
where the $\equiv$ is hiding modding out by the constraints. Of course, this proof involves coefficients of doubly-exponential size, and indeed one can prove that they are required. We will take $\epsilon < 1/2$ for simplicity. Consider a linear functional $\phi: \R[Y] \rightarrow \R$ defined in the following way: any monomial is equivalent to a single power of $y_1$ by the constraints. If that power is $k$, then define $\phi[y_S] = (2\epsilon)^k$. These are the moments that come from the true distribution $y_1 \equiv 2\epsilon$, and $y_i = y_1^{2^{i-1}}$, so $\phi$ is clearly PSD, and it clearly satisfies every constraint except the last one, $y_n^2 = 0$. Assume there is some degree $d$ SOS proof of non-negativity
\[\epsilon - y_1 = \sum_i h_i(y)^2 + \sum_{i=1}^{n-1} \sigma_i(y_i^2 - y_{i+1}) + \lambda \cdot y_n^2,\]
and apply $\phi$ to both sides. We obtain $-\epsilon = P + 0 + \phi[\lambda y_n^2]$, where $P \geq 0$. Because every monomial in $\lambda y_n^2$ gets mapped to a number at most $(2\epsilon)^{2^n}$ and there are at most $O(n^d)$ monomials, $\lambda$ must contain a coefficient of size at least $\Omega(\frac{1}{n^d}\left(\frac{1}{2\epsilon}\right)^{2^n})$. 

Now, the question is how do we modify this example to get a Boolean system that retains the low-degree proof but doesn't gain any proofs with small coefficients. If we simply add the constraints $y_i^2 = y_i$ to the system, then it is easy to prove that $y_i = y_j$ for each $i$ and $j$, and of course $y_n = y_n^2 = 0$. We are making it too easy for SOS to figure out what each $y_i$ should look like. Previously, they were unconstrained in any way, and we want to imitate that. We draw inspiration from the Knapsack problem, and we instead replace each instance of the variable $y_i$ with a sum of $2k$ Boolean variables $\sum_j w_{ij} - k$, and we consider the polynomial $\epsilon - (\sum_j w_{1j} - k)$. Clearly there is a degree two proof of non-negativity for this polynomial since we just replace each instance of $y_i$ with its Boolean sum in the previous proof. It remains to show that there are no other proofs that have only small coefficients. We used Knapsack because it is known that there is no SOS proof of degree less than $\Omega(k)$ that $\sum_j w_{ij} - k$ is not equal to any number $r \in (0,1)$. This allows us to use the Knapsack pseudodistribution to "pretend" that $\sum_j w_{ij} - k = (2\epsilon)^{2^i}$. Specifically, for each $r \in (0,1)$, there is a linear functional $\phi_r$ defined on polynomials of $2k$ Boolean variables which is PSD on monomials up to degree $\Omega(k)$ and for which $\phi_r[\lambda\cdot((\sum_j w_{ij} - k) - r)] = 0$ for any polynomial $\lambda$ up to degree $\Omega(k)$. Now, take the linear functional 
\[\Phi(\R[{\bf w_1}, {\bf w_2}, \dots, {\bf w_n}]) = \phi_{2\epsilon}(\R[{\bf w_1}]) \otimes \phi_{(2\epsilon)^2}(\R[{\bf w_2}])\otimes \dots \otimes \phi_{(2\epsilon)^{2^n}}(\R[{\bf w_n}]).\]
Since each $\phi_r$ is PSD and they are on disjoint variables, $\Phi$ is PSD, and clearly $\Phi$ satisfies all the Boolean constraints on the $w_{ij}$'s. Because $\Phi[\lambda(\sum_j w_{ij} - k)] = \Phi[(2\epsilon)^{2^i}\lambda]$, $\Phi$ also satisfies $\Phi[\lambda((\sum_j w_{ij}-k)^2 - (\sum_j w_{i+1,j} - k))] = 0$ for each $1 \leq i \leq n-1$. Finally, because each variable is Boolean, each monomial is bounded by one, so for any monomial $w_M$, $\Phi[w_M(\sum_j w_{nj} - k)^2] = \Phi[(2\epsilon)^{2^n} w_M] \leq (2\epsilon)^{2^n}$. There are at most $O((nk)^d)$ monomials, so by the same argument for the non-Boolean system, the existence of $\Phi$ implies that any degree $d$ proof of non-negativity for $\epsilon - (\sum_j w_{1j} - k)$ must contain coefficients of size at least $\Omega(\frac{1}{(nk)^d} \cdot \left(\frac{1}{2\epsilon}\right)^{2^n})$.

