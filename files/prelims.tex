\section{Preliminaries}
\label{sec:prelims}
For a set of real polynomials $\cP = \{p_1, p_2, \dots, p_m\}$, we denote their generated ideal in $\R[x]$ by $\langle \cP\rangle$ or $\langle p_1, \dots, p_m\rangle$. We will be working with systems of polynomial constraints, and we will use the $\cP$ to denote the equality constraints, and $\cQ$ to denote the inequality constraints, i.e. $p(x) = 0$ and $q(x) \geq 0$ for $p \in \cP$ and $q \in \cQ$. We will usually use $S$ for the set of points satisfying these constraints. We use $\R[x]_d$ for the set of polynomials of degree at most $d$, and we write ${\bf v}_d$ for the vector of polynomials whose entries are the elements of the usual monomial basis of $\R[x]_d$. Similarly, we use ${\bf v}(\alpha)_d$ for the vector of reals whose entries are the entries of ${\bf v}$ evaluated at $\alpha$. We usually omit the dependencies on $d$ as it is clear from context.
%For an ideal $I$ of $\R[x]$, its real variety is defined 
%\[\cV_\R(I) \coloneqq \{x \in \R | \forall p \in I: p(x) = 0\}.\]
%For a set $S \subseteq \R$, its vanishing ideal is defined
%\[\cI(S) = \{p \in \R[x] | \forall x \in S: p(x) = 0\}.\]
%We call $\cP$ \emph{real complete} if $\cI(\cV_\R(\cP)) = \langle \cP\rangle$. For any polynomial $p \in \langle \cP\rangle$, we say $p$ has a degree $d$ derivation from $\cP$ if there exist there exist real $\lambda_1(x),\dots,\lambda_m(x)$ such that $p(x) = \sum_i \lambda_i(x)p_i(x)$ and $\max_i \deg \lambda_ip_i \leq d$. We often refer to the $\lambda_i$ as the derivation of $p$. Finally, if we call $\cP$ \emph{$k(d)$-effective} if $\cP$ is real complete and any polynomial $p \in \langle \cP\rangle$ of degree $d$ has a derivation from $\cP$ of degree $k(d)$. 
%\begin{lemma}
%If $\cP$ is real complete with finite real variety $S$, then $\R[x]/\langle \cP\rangle$ is an $\R$-vector space of dimension $|S|$ with inner product $\iprod{\overline{p}, \overline{q}}= \sum_{\alpha \in S} p(\alpha)q(\alpha)$ for any $p \in \overline{p}$, $q \in \overline{q}$.
%\end{lemma}
%\begin{proof}
%Let $I = \langle \cP\rangle$. Clearly $\R[x]/I$ is an $\R$-vector space, but furthermore it is easy to show that $\R[x]/I \equiv \R^S$. Consider the evaluation map $E: \R[x]/I \rightarrow \R^S$ given by $E(\overline{p}) = (\alpha \rightarrow p(\alpha))$. First, we show that this map is well-defined. If $p,q \in \overline{p}$, then $p-q \in I$. Because $S$ is the real variety of $\cP$, $(p-q)(\alpha) = 0$ for any $\alpha \in S$, and thus $p(\alpha) = q(\alpha)$. Next, $E$ is clearly linear, and if you take a high enough degree polynomial it is clear that $E$ is onto. Now if $E(\overline{p}) = 0$, then $\forall \alpha \in S: p(\alpha) = 0$. Because $\cP$ is real complete, $p \in I$, and so $\overline{p} = \overline{0}$. The claimed inner product is simply the Euclidean inner product of $\R^S$ composed with the evaluation map $E$. 
%\end{proof}

\subsection{Polynomial Proofs}
Let $\cP = \{p_1,\dots,p_n\}$ be a set of polynomials, and let $S = \{x \in \R^n | \forall p \in \cP: p(x) = 0\}$. We define a proof of membership in $\langle \cP\rangle$ as follows:
\begin{definition}
We say that $r(x)$ has a \emph{derivation} from $\cP$ if there is a polynomial identity of the form
\[r(x) = \sum_{i}^n \lambda_i(x) p_i(x).\]
We say that the proof has degree $d$ if $\max_i \{\deg \lambda_i p_i\} = d$.
\end{definition}

The above proof system is useful for proving when polynomials are zero on $S$, but often we want to prove that they are positive. To that end, let $\cP = \{p_1,\dots,p_n\}$ and $\cQ = \{q_1,\dots,q_m\}$ be two sets of polynomials, and let $S = \{x \in \R^n | \forall p \in \cP: p(x) = 0, \forall q \in \cQ: q(x) \geq 0\}$. We define a proof of non-negativity as follows:
\begin{definition}
 We say that $r(x)$ has a \emph{Sum-of-Squares proof of non-negativity from $\cP$ and $\cQ$} if there is a polynomial identity of the form
\[r(x) = \sum_{i}^{t_0} h_i^2(x) + \sum_{i}^m \left(\sum_{j}^{t_i} s_j^2(x)\right)q_i(x) + \sum_{i}^n \lambda_i(x) p_i(x).\]
We say the proof has degree $d$ if $\max \{\deg h_i^2, \deg s_j^2q_i, \deg \lambda_i p\} = d$.
\end{definition}
The idea behind this terminology is that if such a proof exists, then $r$ must be non-negative on $S$ since the first two terms are non-negative, and the last term is zero. We will be concerned with not just the degree of these proofs, but also their bit complexity. To this end, we define the following norms on polynomials and proofs: For $p(x) \in \R[x]$, we write $\|p\|$ for the absolute value of the maximum coefficient of $p$ in the standard monomial basis, and for any collection of polynomials $\cP$, we write $\|\cP\| = \max_{p \in \cP} \|p\|$. We will also require a bound on $\|S\| = \max_{\alpha \in S} \|\alpha\|$. Throughout this paper we will assume that the solutions $\alpha$ are {\it explicitly bounded} by $\|\alpha\| \leq 2^{\poly(n^d)}$.
%, often referred to as the \emph{explicitly bounded} case. 

\subsection{Rich Solution Spaces}
In this section we define the conditions we require in order to guarantee that SoS proofs from $\cP$ and $\cQ$ have low bit-complexity.
% seemed redundant, commented
%Our goal is to develop easily checkable conditions that suffice to imply that SoS proofs may be taken to lie in a polynomially-bounded region. We will require the constraints to satisfy three properties. 
For a set of assignments $S$ to a polynomial system $(\cP,\cQ)$, define the moment matrix as
\[M_S \coloneqq E_{\alpha \in S}[{\bf v}(\alpha){\bf v}(\alpha)^T]\mcom\]
where the expectation is over the uniform distribution over $S$.  We will omit the subscript and write $M$, if $S$ is clear from the context.

\begin{definition}\label{def:nice}
With the above definitions, 
\begin{itemize}
\item We say that $S$ is \emph{$\delta$-spectrally rich for $(\cP, \cQ)$ up to degree $d$} if every nonzero eigenvalue of $M_S$ is at least $\delta$.
\item We say that $(\cP, \cQ)$ is \emph{$k$-complete on $S$ up to degree $d$} if every zero eigenvector $c$ of $M_S$ (which can be seen as a degree $d$ polynomial $c^T{\bf v}$) has a degree $k$ derivation from $\cP$. 
\item We say that $S$ is \emph{$\epsilon$-robust for $\cQ$} if $\forall q \in \cQ, \forall \alpha \in S: q(\alpha) > \epsilon$.
\end{itemize}
\end{definition}
Spectral richness of the solutions $S$ is equivalent to requiring if $p(x)$ is small on $S$, then there is a polynomial $q$ which agrees with $p$ on $S$ and that has small coefficients. If $(\cP, \cQ, S)$ satisfies all three conditions then we say that $S$ is $(\delta, k, \epsilon)$-\emph{rich} for $(\cP, \cQ)$ up to degree $d$. If $1/\delta = 2^{\poly(n^d)}$, $k = O(d)$, and $1/\epsilon = 2^{\poly(n^d)}$ we simply say $S$ is rich for $(\cP, \cQ)$. We choose these bounds because \prettyref{thm:main} will imply that any constraints with a rich solution space has proofs of non-negativity that can be taken to have polynomial bit complexity. Before we get into the proof of the main theorem, we exhibit polynomial systems that admit rich solutions.

